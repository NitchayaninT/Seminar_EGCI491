\documentclass[a4paper]{article} %Always requireed\part{title}

\usepackage[margin=1.5in]{geometry} %To adjust margins
\usepackage{comment} %To use a block comment in a document
\usepackage{amsmath} % Required to use the equation*
% preamble to declare kackges used in the document
%========================================================================
%Packages that will be used

%To use a block comment in a document
\usepackage{comment} 
\begin{comment}
	Test a block comment 
\end{comment}
%------------------------------------------------------------------------
\usepackage{fontspec}
%Either Xelatex or Lualatex is required as a default complier
%To learn more about "fontspec" package: https://ctan.org/pkg/fontspec?lang=en 
%========================================================================
%To declare a document title and an author(s) 
\title{EGCI491 Computer Engineering Seminar\\ \LaTeX \space Assignment II}
\author{Nitchayanin Thamkunanon}

%document
\begin{document}
	\maketitle
	\setcounter{section} {0}%To start this section with 0
	\section{Mathematical Equations } %Add section

	%Put your text in a body of your document here... 
	\noindent A quadratic equation is a second degree polynomial written as $ax^2+bx+c=0$. The generic function of the quadratic equation can be determined as:
	
	\begin{align}
		\begin{split}
			f^2(x) = ax^2+bx+c\\
		\end{split}
	\end{align}
	\noindent whose discriminant  $b^2-4ac$ is positive, with x representing an unknown, with a, b and c representing constants, and with a $x\not=0$, the quadratic formula is:
	
	\begin{align}
	x = - b\pm\frac{\sqrt{b^2 - 4ac}}{2a}
	\end{align}
	
	\noindent where the plus–minus symbol $\pm$ indicates that the quadratic equation has two solutions. When written separately, they become:
	
		\begin{align}
			\begin{split}
				x &= - b + \frac{\sqrt{b^2 - 4ac}}{2a} \quad and \\
				  x &= - b - \frac{\sqrt{b^2 - 4ac}}{2a}
		     \end{split}
		\end{align}
			
	
\end{document}
%========================================================================